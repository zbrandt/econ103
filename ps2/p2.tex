\begin{homeworkProblem}
    This is a question about grading and course design. There are $N$ students 
    in a course. Assume for simplicity that an instructor can give each student 
    one of two grades, $A$ or $B$.
    
    Say that the class is \emph{curved} if there is a number $x \in \{1, 2, 
    \ldots, N\}$ such that the instructor must give $A$'s to $x$ students, and 
    $B$'s to the remaining $N - x$ students.
    
    \begin{itemize}
        \item[(a)] Suppose the grade in the class is based on homework and 
            exams. To be precise there is a weight $\alpha \in (0, 1)$ such the 
            score of student $i$, denoted by $s_i$, is given by
            \[
                s_i = \alpha h_i + (1 - \alpha)e_i,
            \]
            where $h_i$ is student $i$'s homework score, and $e_i$ is their 
            exam score. If the class is curved, the $x$ students with the 
            highest grade get $A$'s, and the rest get $B$'s (throughout the 
            question you can assume that there are no ties). Fixing the exam 
            scores, suppose the instructor grades homework more leniently. This 
            means that for some $\delta > 1$, each student's grade goes from 
            $h_i$ to $\delta h_i$. What can you say about how the old scores 
            compare to the new scores? What about the new grades? Which 
            students benefit and which are hurt? Be as precise as you can, and 
            prove your answer.
    \end{itemize}
    
    Now, assume instead that each student's score in the course is determined 
    by the amount of time they have to study for the course, $t_i$, and 
    difficulty of the course, $d$, so $s_i = s(t_i, d)$ for some function $s : 
    \mathbb{R}^2_+ \to \mathbb{R}_+$ which is increasing in its first argument 
    (time) and decreasing in the second argument (difficulty).
    \footnote{$\mathbb{R}_+$ denotes the non-negative real numbers.}
    
    Students like getting $A$'s more than $B$'s, but they also like a class 
    that is not too easy, nor too hard (if the class is to easy you don't learn 
    anything new, and if it's too hard you don't understand anything). We can 
    represent student $i$'s preferences over difficulty levels and grades by a 
    utility function
    \[
        U_i(G, d) = u(G) - (d - d_i)^2
    \]
    which describes their payoff from getting grade $G \in \{A, B\}$ when the 
    class has difficulty level $d \in \mathbb{R}_+$. The parameter $d_i$ 
    represents the ideal difficulty level for student $i$, if they had no 
    concern about grades (make sure you understand why this is the case).
    
    The instructor wants to choose the difficulty level efficiently, to 
    maximize the sum of student utilities.
    
    \begin{itemize}
        \item[(b)] If the class is curved, what is the efficient choice of $d$?
    \end{itemize}
    
    In reality the instructor does not observe $(t_i, d_i)^N_{i=1}$. Instead, 
    each student privately knows their own $t_i$ and $d_i$. In order to set the 
    level of difficulty for the course the instructor has to run a survey to 
    learn about $\langle d_i \rangle^N_{i=1}$. The instructor is worried about 
    students' incentives to report truthfully.
    
    Assume that $t_i$ and $d_i$ are fixed for each student, and will remain the 
    same regardless of what difficulty level is chosen.\footnote{In particular,  
    students cannot adjust the time devoted to the course as a function of the 
    course difficulty. This is obviously a simplification of reality. It turns 
    out, however, that it doesn't really affect the conclusions here.} However 
    students can report whatever $t_i, d_i$ they want to the instructor.
    
    \begin{itemize}
        \item[(c)] Suppose that the class is curved as in 2(b). If the 
            instructor wants to implement the efficient choice of $d$, is it a 
            dominant strategy for all students to report truthfully? If not, in 
            which direction would students like to misreport? Prove your answer.
        
        \item[(d)] Now assume, as before, that the class is curved. Assume also 
        that $U_i(G, d) = u(G) - |d - d_i|$. Prove that it is indeed a dominant 
        strategy for students to report truthfully.
    \end{itemize}

    \pagebreak
    \part

    Suppose the grade in the class is based on homework and exams. To be 
    precise there is a weight $\alpha \in (0, 1)$ such the score of student 
    $i$, denoted by $s_i$, is given by
    \[
        s_i = \alpha h_i + (1 - \alpha)e_i,
    \]
    where $h_i$ is student $i$'s homework score, and $e_i$ is their exam score. 
    If the class is curved, the $x$ students with the highest grade get $A$'s, 
    and the rest get $B$'s (throughout the question you can assume that there 
    are no ties). Fixing the exam scores, suppose the instructor grades 
    homework more leniently. This means that for some $\delta > 1$, each 
    student's grade goes from $h_i$ to $\delta h_i$. What can you say about how 
    the old scores compare to the new scores? What about the new grades? Which 
    students benefit and which are hurt? Be as precise as you can, and prove 
    your answer.

    \solution \\
    All students' scores increase. Specifically, the new score is
    
    \[
        s_i' = \alpha(\delta h_i) + (1 - \alpha)e_i = \alpha h_i + (1-\alpha)e_i 
        + \alpha(\delta - 1)h_i = s_i + \alpha(\delta - 1)h_i > s_i
    \]
    
    The increase in score is $\alpha(\delta - 1)h_i$, which is proportional to 
    the homework score. Consider two students $i$ and $j$, their pairwise
    difference is

    \[
        s_i' - s_j' = (s_i - s_j) + \alpha (\delta - 1) (h_i - h_j)
    \]

    If $h_i \geq h_j$, then $\alpha (\delta - 1)(h_i - h_j) \geq 0$, so $s_i' -
    s_j' \geq s_i - s_j$. Increasing the weight on homework increases the 
    advantage of the student with the higher homework score. Any pair whose
    homework and exam order agree, e.g., $h_i \geq h_j$ and $e_i \geq e_j$,
    keeps the same ordering after the homework weight change. If $s_i > s_j$
    and $h_i \geq h_j$, then $s_i' > s_j'$. Therefore, only those pairs of 
    students for which homework and exam rankings point in opposing directions 
    can change their relative ordering. The set of top-$x$ students can change
    only by substituing students who are relatively strong on homework for
    students who are relatively weak on homework. Students with relatively
    large homework scores, relative to other students with similar exam scores,
    gain relative and are more likely to move into the top $x$. Students with 
    relatively small homework scores lose relative rank. 

    \pagebreak

    \part

    If the class is curved, what is the efficient choice of $d$?

    \solution \\
    The instructor maximizes the sum of utilities:
    \[
        \max_d \sum_{i=1}^N U_i(G_i, d) = \max_d \sum_{i=1}^N \left[u(G_i) - 
        (d - d_i)^2\right]
    \]

    Since the class is curved, exactly $x$ students receive grade $A$ and $N-x$ 
    students receive grade $B$, determined by their ranking based on 
    $s(t_i, d)$. For any given $d$, the grades $\{G_i\}_{i=1}^N$ are 
    determined, so we can focus on minimizing the difficulty cost:
    \[
        \min_d \sum_{i=1}^N (d - d_i)^2
    \]

    Taking the first-order condition:
    \[
        \frac{\partial}{\partial d}\sum_{i=1}^N (d - d_i)^2 = 2\sum_{i=1}^N 
        (d - d_i) = 0
    \]

    Solving for $d$:
    \[
        Nd = \sum_{i=1}^N d_i \implies d^* = \frac{1}{N}\sum_{i=1}^N d_i
    \]

    The second-order condition is $2N > 0$, confirming this is a minimum. The 
    efficient difficulty level is the arithmetic mean of all students' ideal 
    difficulty levels: $d^* = \bar{d} = \frac{1}{N} \sum_{i=1}^N d_i$.

    \pagebreak

    \part

    Suppose that the class is curved as in 2(b). If the instructor wants to 
    implement the efficient choice of $d$, is it a dominant strategy for all 
    students to report truthfully? If not, in which direction would students 
    like to misreport? Prove your answer.

    \solution \\
    The instructor, lacking direct observation of students' $d_i$'s, uses the 
    reported $\hat{d}_1, \dots, \hat{d}_N$ and sets

    \[
        \hat{d} = \frac{1}{N} \sum_{i=1}^{N} \hat{d}_i
    \]

    Student $i$'s utility is
    
    \[
        U_i(G_i, \hat{d}) = u(G_i) - (\hat{d} - d_i)^2 = u(G_i) - \left( \frac{\hat{d}_i + 
        \sum_{j \neq i} \hat{d}_j} {N} - d_i \right)^2
    \]

    Minimizing $\left( \frac{\hat{d}_i + \sum_{j \neq i} \hat{d}_j} {N} - d_i 
    \right)^2$ with respect to $\hat{d}_i$ gives the best response for student
    $i$ to report to the instructor

    \[
        \hat{d}_i^{\text{BR}} = N d_i - \sum_{j \neq i} \hat{d}_j
    \]

    Therefore, truthfully reporting $\hat{d}_i = d_i$ is generally not the best
    response and not a dominant strategy. A student can move the chosen $d$ 
    closer to their ideal by misreporting their preferences. If the average of
    the other students' reports is below $d_i$, student $i$ will report a value
    larger than $d_i$, to raise $\hat{d}$, and vice versa. 


    \pagebreak

    \part

    Now assume, as before, that the class is curved. Assume also that 
    $U_i(G, d) = u(G) - |d - d_i|$. Prove that it is indeed a dominant 
    strategy for students to report truthfully.

    \solution \\
    If $U_i(G, d)$ now equals $u(G) - |d - d_i|$, then the $d$ that maximizes 
    the social objective is the median of the reported $\hat{d}_i$. I'll show
    that it is a dominant strategy for students to report truthfully in this 
    case.

    For a student $i$, the reports of all other students are $\hat{d}_{-i}$

    \[
        \hat{d}_{-i} = \{ \hat{d}_1, \dots \hat{d}_{i - 1}, \hat{d}_{i + 1}, \dots, \hat{d}_N \}
    \]

    Student $i$'s optimization problem is to find the right $\hat{d}_i$ to
    report to maximize their utility. Since the instructor chooses the median
    of the reported difficulties, the student $i$ needs to choose $\hat{d}_i$
    such that the median is as close as possible to their preferred difficulty
    $d_i$. There are three cases to consider:

    \begin{enumerate}
        \item The median of $\hat{d}_{-i}$ is less than $d_i$: in this case, 
            reporting truthfully weakly dominates since, unlike in the 
            previous part, scaling the magnitude higher in an attempt to move
            the median to student $i$'s advantage will have no greater effect 
            than reporting $d_i$ since the median is resistant to extreme 
            outliers. 
        \item The median of $\hat{d}_{-i}$ is equal to $d_i$: in this case,
            there is no benefit to not reporting truthfully, since reporting
            $d_i$ will cement this value as the median and optimize student 
            $i$'s utility.
        \item The median of $\hat{d}_{-i}$ is greater than $d_i$: this case is
            similar to the first one, except in the opposite direction. Again, 
            reporting truthfully weakly dominates since scaling the magnitude
            of $\hat{d}_i$ lower will have no greater effect than reporting 
            $d_i$ since the median is resistant to extreme outliers. 
    \end{enumerate}

    So, reporting truthfully is indeed the dominant strategy with this
    configuration. 

\end{homeworkProblem}
