\begin{homeworkProblem}
    A group of people met and some of them shook each-other's hands. Prove that
    the number of people who shook others' hands an odd number of times is, in
    fact, even.
    \\

    \solution

    \begin{proof}
    
    This situtation can be represented in the form of a graph, where people are
    vertices and handshakes are edges. The degree of any vertex corresponds to
    the number of handshakes a person made. The sum of the degrees of all
    vertices add up to twice the number of edges there are in the graph. This
    is because each new edge increases the degrees of two vertices, since an 
    edge must connect two vertices. 
    
    \[
        \sum_{v \in V} \deg v = 2 |E| 
    \]

    Since the right-hand side of the equation, two times the number of 
    vertices, is a multiple of two, the number is even. The left-hand side sum 
    of the above equation can be decomposed into a sum over all the vertices 
    with odd degrees and a sum over all the vertices with even degrees. 

    \[
        \sum_{v_{\text{odd}} \in V} \deg v + \sum_{v_{\text{even}} \in V} \deg v = 2 |E|
    \]

    The sum of degrees over all even-degree vertices is naturally even. The sum
    of degrees over all odd-degree vertices must then also be even since $2|E|$
    is an even number. For $\sum \deg v_{\text{odd}}$ to be even, there must be
    an even number of odd-degree vertices. Since this formulation corresponds
    one-to-one with people and handshakes, this shows that the number of people
    in a group who shake others' hands an odd number of times is even. 
    \end{proof}

\end{homeworkProblem}